\documentclass[a4paper]{article}
\usepackage{graphicx} % Required for inserting images
\usepackage{amsmath}
\usepackage{amsfonts, amssymb}
\usepackage{amsthm}
\usepackage{amsrefs}
\usepackage[margin=1in]{geometry}
\usepackage{mathtools}
\usepackage{microtype}
\usepackage{xcolor}
\usepackage{hyperref}
\hypersetup{
    citecolor=black,
    citebordercolor=orange,
    colorlinks=true,
    linkcolor=black,
    filecolor=magenta,      
    urlcolor=cyan,
    pdftitle={Factorization of Complex Numbers in the Form a+bi over Quaternion Algebras},
    pdfauthor={Cédric Skwar},
    pdfpagemode=FullScreen,
}

\usepackage{fancyhdr}
\pagestyle{fancy}

\lhead{Cédric Skwar}
\rhead{Page \thepage}

\title{Factorization of Complex Numbers in the Form \(a + bi\) over Quaternion Algebras}
\author{Cédric Skwar}
\date{October 2023}

\begin{document}

\maketitle

\section{Definitions}
Quaternions are an extension of the complex numbers, defined as \( q = a + bi + cj + dk \), where \( a, b, c, d \) are real numbers and \( i, j, k \) are the quaternion units, as introduced by Hamilton \cite{hamilton1847elements}. Unlike real or complex numbers, quaternions are noncommutative, meaning \( xy \neq yx \) in general \cite{conway1997quaternions}. 
In terms of properties, quaternions are associative and distribute over addition, but they lack commutativity for multiplication \cite{conway1997quaternions}. They can be understood as a generalization of complex numbers, which can be represented as quaternions with \( c = d = 0 \) \cite{kuipers1999quaternions}.
\begin{table}[h]
    \centering
    \begin{tabular}{ccccc}
        \( \downarrow \rightarrow \) & \( 1 \) & \( i \) & \( j \) & \( k \) \\
        \( 1 \) & \( 1 \) & \( i \) & \( j \) & \( k \) \\
        \( i \) & \( i \) & \( -1 \) & \( k \) & \( -j \) \\
        \( j \) & \( j \) & \( -k \) & \( -1 \) & \( i \) \\
        \( k \) & \( k \) & \( j \) & \( -i \) & \( -1 \) \\
    \end{tabular}
    \caption{Quaternion multiplication table. Note the noncommutative nature; for example, \( ij = k \) but \( ji = -k \) \cite{ward1997quaternions}.}
    \label{tab:mult_table}
\end{table}

\section{Axioms}
In this section, we introduce the axioms for the binary operation \(*\). We begin with the Axiom of Associativity:
\[
(x*y)*z=x*(y*z).
\]
Next, we state the Identity Axiom, where \(1\) denotes the identity element for the operation \(*\):
\[
1*x=x*1=x.
\]


\section{Proof}

\section{Final words}

\bibliography{bib/introduction}

\end{document}

